\documentclass[11pt,a4paper]{article}
\usepackage[utf8]{inputenc}
\usepackage[margin=1in]{geometry}
\usepackage{graphicx}
\usepackage{hyperref}
\usepackage{longtable}
\usepackage{array}
\usepackage{booktabs}
\usepackage{fancyhdr}
\usepackage{xcolor}
\usepackage{enumitem}
\usepackage{amsmath}
\usepackage{amssymb}
\usepackage{textcomp}

% Header and footer
\pagestyle{fancy}
\fancyhf{}
\fancyhead[L]{MMM Trainer Deployment Guide}
\fancyhead[R]{\thepage}
\fancyfoot[C]{Confidential}

% Colors
\definecolor{darkblue}{RGB}{0,51,102}
\definecolor{lightgray}{RGB}{240,240,240}

% Title page
\title{\Huge\textbf{MMM Trainer\\Deployment Guide}\\[0.5cm]\Large Self-Hosted Marketing Mix Modeling Platform}
\author{Deployment Documentation for Customers}
\date{\today}

\begin{document}

\maketitle
\thispagestyle{empty}

\newpage
\tableofcontents
\newpage

\section{Executive Summary}

The MMM Trainer is a cloud-native Marketing Mix Modeling platform built on Google Cloud Platform (GCP). It provides a web-based interface for data scientists and analysts to run Facebook's Robyn MMM framework at scale, processing data from Snowflake and delivering production-ready model insights.

\subsection{Key Features}

\begin{itemize}
    \item \textbf{Web-Based Interface:} Streamlit application accessible via browser
    \item \textbf{Snowflake Integration:} Direct connection to your data warehouse
    \item \textbf{Robyn MMM Engine:} Industry-standard R/Robyn framework
    \item \textbf{Cloud-Native Architecture:} Scalable, secure, and cost-effective
    \item \textbf{Batch Processing:} Queue system for running multiple experiments
    \item \textbf{Results Management:} Automated storage and visualization of model outputs
\end{itemize}

\subsection{Infrastructure Overview}

The platform consists of:
\begin{itemize}
    \item \textbf{Web Service:} Streamlit application on Cloud Run (always-on)
    \item \textbf{Training Jobs:} R/Robyn execution on Cloud Run Jobs (on-demand)
    \item \textbf{Storage:} Google Cloud Storage for data and model artifacts
    \item \textbf{Scheduler:} Cloud Scheduler for batch job processing
    \item \textbf{Secrets:} Google Secret Manager for credential management
\end{itemize}

\newpage

\section{Cost Estimates}

All costs are based on actual production measurements using Google Cloud Platform services in the \texttt{europe-west1} region. Costs are billed per-second for compute resources.

\subsection{Training Job Performance}

\begin{table}[h]
\centering
\begin{tabular}{lrrrl}
\toprule
\textbf{Workload} & \textbf{Iterations × Trials} & \textbf{Duration} & \textbf{Cost/Job} & \textbf{Use Case} \\
\midrule
Test Run & 200 × 3 = 600 & 0.8 min & \$0.01 & Quick validation \\
\textbf{Benchmark} & \textbf{2,000 × 5 = 10K} & \textbf{12 min} & \textbf{\$0.20} & \textbf{Standard testing} \\
\textbf{Production} & \textbf{10,000 × 5 = 50K} & \textbf{80-120 min} & \textbf{\$1.33-\$2.00} & \textbf{Production runs} \\
Large Production & 10,000 × 10 = 100K & 160-240 min & \$2.67-\$4.00 & High-quality runs \\
\bottomrule
\end{tabular}
\caption{Training job durations and costs (8 vCPU, 32GB RAM configuration)}
\end{table}

\textbf{Note:} Benchmark runs (12-18 minutes) are ideal for experimentation and testing. Production runs (80-120 minutes) provide more thorough model exploration and are recommended for final models.

\subsection{Monthly Cost Scenarios}

\begin{table}[h]
\centering
\begin{tabular}{lrrrr}
\toprule
\textbf{Usage Level} & \textbf{Web Calls} & \textbf{Training Jobs} & \textbf{Benchmark Cost} & \textbf{Production Cost} \\
\midrule
Light & 100 & 10 & \$4 & \$15-22 \\
Moderate & 500 & 50 & \$12 & \$69-102 \\
Heavy & 1,000 & 100 & \$22 & \$135-202 \\
Very Heavy & 5,000 & 500 & \$102 & \$667-1002 \\
\bottomrule
\end{tabular}
\caption{Monthly cost estimates by usage volume}
\end{table}

\subsection{Cost Breakdown}

\textbf{Fixed Monthly Costs} (\textasciitilde\$2/month):
\begin{itemize}
    \item GCS storage: \$0.50-\$2.00 (depends on data retention)
    \item Secret Manager: \$0.36 (6 secrets × \$0.06)
    \item Cloud Scheduler: \$0.30 (covered by free tier)
    \item Artifact Registry: \$0.50 (container images)
\end{itemize}

\textbf{Variable Costs (per usage):}
\begin{itemize}
    \item Training jobs: \$0.20 (benchmark) to \$1.33-\$2.00 (production) per job
    \item Web service: \textasciitilde\$0.002 per request (negligible)
    \item Snowflake: Separate (depends on warehouse size and queries)
\end{itemize}

\textbf{Key Cost Drivers:}
\begin{itemize}
    \item Training jobs account for 90-99\% of total costs at scale
    \item Faster machines have similar costs due to per-second billing
    \item Storage costs are minimal with lifecycle policies
    \item Network egress charged only for downloads outside GCP region
\end{itemize}

\newpage

\section{Minimum Technical Requirements}

\subsection{Required Skills and Knowledge}

The technical team maintaining this application should have:

\subsubsection{Essential Skills}
\begin{itemize}
    \item \textbf{Google Cloud Platform:}
    \begin{itemize}
        \item Basic understanding of Cloud Run, GCS, and IAM
        \item Ability to navigate GCP Console
        \item Understanding of billing and cost management
    \end{itemize}
    \item \textbf{Infrastructure as Code:}
    \begin{itemize}
        \item Basic Terraform knowledge for infrastructure changes
        \item Ability to read and modify Terraform configuration files
    \end{itemize}
    \item \textbf{Version Control:}
    \begin{itemize}
        \item Git and GitHub workflows
        \item Understanding of CI/CD concepts
    \end{itemize}
    \item \textbf{Container Technology:}
    \begin{itemize}
        \item Basic Docker concepts
        \item Understanding of container registries
    \end{itemize}
\end{itemize}

\subsubsection{Recommended Skills}
\begin{itemize}
    \item \textbf{Programming Languages:}
    \begin{itemize}
        \item Python (for Streamlit application modifications)
        \item R (for Robyn MMM customizations)
    \end{itemize}
    \item \textbf{Data Warehouse:}
    \begin{itemize}
        \item Snowflake query optimization
        \item SQL for data extraction
    \end{itemize}
    \item \textbf{Monitoring and Debugging:}
    \begin{itemize}
        \item Cloud Logging for troubleshooting
        \item Performance monitoring and optimization
    \end{itemize}
\end{itemize}

\subsection{Required Tools}

All team members should have access to:

\begin{table}[h]
\centering
\begin{tabular}{lll}
\toprule
\textbf{Tool} & \textbf{Version} & \textbf{Purpose} \\
\midrule
Google Cloud SDK & Latest & GCP CLI operations \\
Terraform & $\geq$ 1.5.0 & Infrastructure management \\
Docker & Latest & Container testing (optional) \\
Git & Latest & Version control \\
Python & $\geq$ 3.11 & Local development (optional) \\
\bottomrule
\end{tabular}
\caption{Required development tools}
\end{table}

\subsection{Access Requirements}

\subsubsection{Google Cloud Platform}
\begin{itemize}
    \item \textbf{For Monitoring:} Viewer role
    \item \textbf{For Deployments:} Editor or specific roles:
    \begin{itemize}
        \item Cloud Run Admin
        \item Storage Admin
        \item Secret Manager Admin
        \item Service Account Admin
    \end{itemize}
    \item \textbf{For Debugging:} Logs Viewer, Monitoring Viewer
\end{itemize}

\subsubsection{GitHub Repository}
\begin{itemize}
    \item \textbf{For Development:} Write access
    \item \textbf{For Releases:} Maintain or Admin access
    \item \textbf{For Secrets Management:} Admin access
\end{itemize}

\subsubsection{Snowflake}
\begin{itemize}
    \item Read access to source data tables
    \item Access to a dedicated warehouse for queries
    \item Appropriate role (not ACCOUNTADMIN in production)
\end{itemize}

\newpage

\section{Infrastructure Requirements}

\subsection{Google Cloud Platform Resources}

\subsubsection{Required GCP APIs}
The following APIs must be enabled in your GCP project:

\begin{itemize}
    \item \texttt{run.googleapis.com} - Cloud Run
    \item \texttt{artifactregistry.googleapis.com} - Container registry
    \item \texttt{cloudbuild.googleapis.com} - Container builds
    \item \texttt{cloudscheduler.googleapis.com} - Job scheduling
    \item \texttt{secretmanager.googleapis.com} - Secrets management
    \item \texttt{storage.googleapis.com} - Cloud Storage
    \item \texttt{iamcredentials.googleapis.com} - Service accounts
    \item \texttt{iam.googleapis.com} - IAM management
    \item \texttt{cloudresourcemanager.googleapis.com} - Project management
\end{itemize}

\subsubsection{Required GCP Resources}

\begin{table}[h]
\centering
\begin{tabular}{lll}
\toprule
\textbf{Resource} & \textbf{Purpose} & \textbf{Example Name} \\
\midrule
GCP Project & Container for all resources & \texttt{mmm-production} \\
GCS Bucket (State) & Terraform state storage & \texttt{project-tf-state} \\
GCS Bucket (App) & Model outputs \& data & \texttt{mmm-app-output} \\
Artifact Registry & Docker image storage & \texttt{mmm-repo} \\
Workload Identity Pool & GitHub authentication & \texttt{github-pool} \\
Service Account (Deploy) & CI/CD deployment & \texttt{github-deployer} \\
Service Account (Web) & Web service runtime & \texttt{mmm-web-sa} \\
Service Account (Training) & Training job runtime & \texttt{mmm-training-sa} \\
\bottomrule
\end{tabular}
\caption{Required GCP resources}
\end{table}

\subsection{Compute Resources}

\subsubsection{Web Service Configuration}
\begin{itemize}
    \item \textbf{Platform:} Cloud Run Service
    \item \textbf{CPU:} 2 vCPU
    \item \textbf{Memory:} 4GB RAM
    \item \textbf{Min Instances:} 0 (cost-optimized) or 1+ (low latency)
    \item \textbf{Max Instances:} 10 (adjustable)
    \item \textbf{Timeout:} 60 seconds per request
    \item \textbf{Concurrency:} 80 requests per instance
\end{itemize}

\subsubsection{Training Job Configuration}
\begin{itemize}
    \item \textbf{Platform:} Cloud Run Jobs
    \item \textbf{CPU:} 8 vCPU (recommended for 12-minute runs)
    \item \textbf{Memory:} 32GB RAM
    \item \textbf{Timeout:} 6 hours (adjustable)
    \item \textbf{Max Retries:} 1
    \item \textbf{Parallelism:} 1 (single job at a time)
\end{itemize}

\textbf{Note:} CPU/memory can be adjusted in Terraform configuration. More cores = faster training but similar cost due to per-second billing.

\subsection{Storage Requirements}

\subsubsection{Google Cloud Storage}
\begin{itemize}
    \item \textbf{Estimated Usage:}
    \begin{itemize}
        \item Base: 80GB for historical data
        \item Growth: \textasciitilde2GB per training run
    \end{itemize}
    \item \textbf{Lifecycle Policies:}
    \begin{itemize}
        \item Move to Nearline after 30 days (50\% cost reduction)
        \item Move to Coldline after 90 days (80\% cost reduction)
    \end{itemize}
    \item \textbf{Backup:} Enabled via versioning
\end{itemize}

\subsection{Network Requirements}

\begin{itemize}
    \item \textbf{Outbound:} HTTPS access to:
    \begin{itemize}
        \item Snowflake data warehouse
        \item GitHub (for CI/CD)
        \item Docker Hub / Package registries
    \end{itemize}
    \item \textbf{Inbound:} HTTPS only (TLS 1.2+)
    \item \textbf{Domain:} Optional custom domain via Cloud Run
    \item \textbf{Firewall:} Managed by GCP (no configuration needed)
\end{itemize}

\subsection{Security Requirements}

\subsubsection{Authentication}
\begin{itemize}
    \item \textbf{User Authentication:} Google OAuth 2.0
    \begin{itemize}
        \item Domain-restricted access (e.g., @company.com)
        \item OAuth consent screen configuration
        \item Client ID and Client Secret
    \end{itemize}
    \item \textbf{Service Authentication:} GCP service accounts
    \item \textbf{Snowflake Authentication:} RSA private key (preferred) or password
\end{itemize}

\subsubsection{Secrets Management}
\begin{itemize}
    \item All secrets stored in Google Secret Manager
    \item Automatic secret rotation supported
    \item No secrets in code or environment variables
    \item Audit logging for secret access
\end{itemize}

\subsubsection{IAM Permissions}
Service accounts follow principle of least privilege:
\begin{itemize}
    \item \textbf{Web Service:} Artifact Registry Reader, Storage Object Admin, Secret Accessor
    \item \textbf{Training Job:} Storage Object Admin, Secret Accessor
    \item \textbf{Deployer:} Cloud Run Admin, Storage Admin, Service Account User
\end{itemize}

\newpage

\section{Deployment Overview}

\subsection{Deployment Architecture}

The application uses GitHub Actions for continuous deployment with Terraform managing infrastructure as code.

\subsubsection{Deployment Flow}
\begin{enumerate}
    \item Developer pushes code to GitHub repository
    \item GitHub Actions triggered automatically
    \item Docker images built (web service + training job)
    \item Images pushed to Artifact Registry
    \item Terraform applies infrastructure changes
    \item Cloud Run service and jobs updated
    \item Health checks verify deployment
\end{enumerate}

\subsubsection{Environments}
\begin{itemize}
    \item \textbf{Production:} Deployed from \texttt{main} branch
    \item \textbf{Development:} Deployed from \texttt{dev} or \texttt{feat-*} branches
    \item Separate Terraform workspaces for isolation
\end{itemize}

\subsection{Initial Setup Steps}

\subsubsection{Phase 1: GCP Project Setup (1-2 hours)}
\begin{enumerate}
    \item Create or select GCP project
    \item Enable required APIs
    \item Create Terraform state bucket
    \item Set up billing alerts
\end{enumerate}

\subsubsection{Phase 2: Authentication Setup (1-2 hours)}
\begin{enumerate}
    \item Configure Workload Identity Federation for GitHub
    \item Create service accounts with appropriate roles
    \item Set up Google OAuth consent screen and credentials
    \item Generate or obtain Snowflake RSA key pair
\end{enumerate}

\subsubsection{Phase 3: Infrastructure Deployment (30 minutes)}
\begin{enumerate}
    \item Create Artifact Registry repository
    \item Create GCS bucket for outputs
    \item Configure Terraform backend and variables
    \item Update GitHub repository secrets
\end{enumerate}

\subsubsection{Phase 4: Application Deployment (15 minutes)}
\begin{enumerate}
    \item Push code to trigger CI/CD
    \item Monitor GitHub Actions workflow
    \item Verify Cloud Run service deployment
    \item Test web interface access
\end{enumerate}

\subsubsection{Phase 5: Verification (30 minutes)}
\begin{enumerate}
    \item Test Snowflake connection
    \item Run benchmark training job
    \item Verify results in GCS
    \item Check logs and monitoring
\end{enumerate}

\textbf{Total Estimated Time:} 3-5 hours for initial deployment

\subsection{Configuration Files}

Key configuration files to customize:

\begin{itemize}
    \item \texttt{infra/terraform/backend.tf} - Terraform state bucket
    \item \texttt{infra/terraform/envs/prod.tfvars} - Production settings
    \item \texttt{infra/terraform/envs/dev.tfvars} - Development settings
    \item \texttt{.github/workflows/ci.yml} - Production CI/CD
    \item \texttt{.github/workflows/ci-dev.yml} - Development CI/CD
\end{itemize}

\subsection{Required Credentials}

Collect these credentials before deployment:

\begin{table}[h]
\centering
\small
\begin{tabular}{lp{8cm}}
\toprule
\textbf{Credential} & \textbf{Description} \\
\midrule
GCP Project ID & Your Google Cloud project identifier \\
GCP Project Number & Numeric project identifier (auto-generated) \\
OAuth Client ID & From GCP OAuth consent screen \\
OAuth Client Secret & From GCP OAuth consent screen \\
Cookie Secret & Random 32-byte hex (generate with \texttt{openssl rand -hex 32}) \\
SF Account & Snowflake account identifier \\
SF User & Snowflake username \\
SF Private Key & RSA private key in PEM format (preferred) \\
SF Warehouse & Snowflake warehouse name \\
SF Database & Snowflake database name \\
SF Schema & Snowflake schema name \\
\bottomrule
\end{tabular}
\caption{Required credentials for deployment}
\end{table}

\newpage

\section{Ongoing Maintenance}

\subsection{Regular Maintenance Tasks}

\subsubsection{Daily/Weekly}
\begin{itemize}
    \item Monitor Cloud Run service health
    \item Check training job success rates
    \item Review logs for errors or warnings
    \item Monitor GCS storage growth
\end{itemize}

\subsubsection{Monthly}
\begin{itemize}
    \item Review GCP billing and optimize costs
    \item Analyze Snowflake warehouse usage
    \item Clean up old training artifacts (automated via lifecycle)
    \item Review and update access permissions
\end{itemize}

\subsubsection{Quarterly}
\begin{itemize}
    \item Update Python dependencies (security patches)
    \item Update R packages for Robyn
    \item Review and optimize Cloud Run resources
    \item Test disaster recovery procedures
    \item Rotate sensitive credentials (OAuth, Snowflake keys)
\end{itemize}

\subsection{Monitoring and Alerting}

\subsubsection{Key Metrics to Monitor}
\begin{itemize}
    \item \textbf{Web Service:}
    \begin{itemize}
        \item Request latency (target: <2s)
        \item Error rate (target: <1\%)
        \item CPU/Memory utilization
    \end{itemize}
    \item \textbf{Training Jobs:}
    \begin{itemize}
        \item Job success rate (target: >95\%)
        \item Average execution time
        \item Resource utilization
    \end{itemize}
    \item \textbf{Storage:}
    \begin{itemize}
        \item Storage growth rate
        \item Bucket size vs lifecycle policies
    \end{itemize}
    \item \textbf{Costs:}
    \begin{itemize}
        \item Monthly spend by service
        \item Cost per training job
        \item Storage costs
    \end{itemize}
\end{itemize}

\subsubsection{Recommended Alerts}
\begin{itemize}
    \item Budget alert at 50\% and 90\% of monthly budget
    \item Error rate exceeds 5\% for 5 minutes
    \item Training job failures exceed 10\%
    \item Storage exceeds 500GB (adjust based on usage)
\end{itemize}

\subsection{Backup and Disaster Recovery}

\subsubsection{What is Backed Up}
\begin{itemize}
    \item \textbf{Code:} Version controlled in GitHub
    \item \textbf{Infrastructure:} Defined in Terraform (version controlled)
    \item \textbf{Terraform State:} Stored in GCS with versioning enabled
    \item \textbf{Model Artifacts:} Stored in GCS with retention policies
    \item \textbf{Secrets:} Managed in Google Secret Manager with versions
\end{itemize}

\subsubsection{Recovery Time Objectives}
\begin{itemize}
    \item \textbf{Web Service:} <15 minutes (redeploy from GitHub)
    \item \textbf{Infrastructure:} <30 minutes (Terraform apply)
    \item \textbf{Data Loss:} Minimal (GCS has 11 nines durability)
\end{itemize}

\subsubsection{Disaster Recovery Procedure}
\begin{enumerate}
    \item Create new GCP project (if primary is unavailable)
    \item Run initial setup (APIs, buckets, service accounts)
    \item Deploy infrastructure via Terraform
    \item Trigger CI/CD deployment from GitHub
    \item Restore secrets from backup documentation
    \item Verify functionality and notify users
\end{enumerate}

\subsection{Security Maintenance}

\subsubsection{Credential Rotation Schedule}
\begin{itemize}
    \item \textbf{Snowflake Private Key:} Annually or upon personnel changes
    \item \textbf{OAuth Secrets:} Annually
    \item \textbf{Service Account Keys:} Not needed (uses Workload Identity)
    \item \textbf{Cookie Secret:} Annually or after security incidents
\end{itemize}

\subsubsection{Access Review}
\begin{itemize}
    \item Review GCP IAM permissions quarterly
    \item Remove access for departed team members immediately
    \item Audit service account permissions annually
    \item Review Google OAuth allowed domains
\end{itemize}

\subsubsection{Security Updates}
\begin{itemize}
    \item Apply Python security updates within 30 days
    \item Update base Docker images quarterly
    \item Monitor GitHub security advisories
    \item Subscribe to GCP security bulletins
\end{itemize}

\subsection{Troubleshooting Common Issues}

\subsubsection{Web Service Not Responding}
\begin{enumerate}
    \item Check Cloud Run service status in GCP Console
    \item Review logs in Cloud Logging
    \item Verify service account permissions
    \item Check OAuth configuration
    \item Restart service if needed (redeploy from GitHub)
\end{enumerate}

\subsubsection{Training Jobs Failing}
\begin{enumerate}
    \item Check job execution logs in Cloud Logging
    \item Verify Snowflake connectivity and credentials
    \item Check GCS bucket permissions
    \item Review memory/CPU allocation (increase if OOM errors)
    \item Examine \texttt{robyn\_console.log} in GCS results folder
\end{enumerate}

\subsubsection{High Costs}
\begin{enumerate}
    \item Review Cloud Run billing in GCP Console
    \item Check number of training jobs executed
    \item Verify min\_instances setting (should be 0 for cost optimization)
    \item Review GCS storage lifecycle policies
    \item Analyze Snowflake warehouse usage
\end{enumerate}

\subsection{Update Procedures}

\subsubsection{Updating Application Code}
\begin{enumerate}
    \item Create feature branch from \texttt{dev}
    \item Make and test changes locally
    \item Push to GitHub (triggers dev deployment)
    \item Test in dev environment
    \item Create pull request to \texttt{main}
    \item Review and merge (triggers production deployment)
\end{enumerate}

\subsubsection{Updating Infrastructure}
\begin{enumerate}
    \item Modify Terraform files in \texttt{infra/terraform/}
    \item Test changes in dev environment first
    \item Review Terraform plan output carefully
    \item Apply changes via GitHub Actions
    \item Verify resources in GCP Console
\end{enumerate}

\subsubsection{Updating Dependencies}
\begin{enumerate}
    \item Update \texttt{requirements.txt} (Python) or \texttt{Dockerfile.training-base} (R)
    \item Test locally or in dev environment
    \item Check for breaking changes in package release notes
    \item Deploy to dev, verify functionality
    \item Deploy to production after successful testing
\end{enumerate}

\newpage

\section{Support and Resources}

\subsection{Documentation}

\begin{itemize}
    \item \textbf{README.md} - Project overview and quick start
    \item \textbf{ARCHITECTURE.md} - System architecture and components
    \item \textbf{DEVELOPMENT.md} - Local development setup
    \item \textbf{REQUIREMENTS.md} - Detailed technical requirements
    \item \textbf{DEPLOYMENT\_GUIDE.md} - Step-by-step deployment instructions
    \item \textbf{COST\_OPTIMIZATION.md} - Cost management strategies
\end{itemize}

\subsection{External Resources}

\begin{itemize}
    \item \textbf{Google Cloud Documentation:} \url{https://cloud.google.com/docs}
    \item \textbf{Cloud Run Pricing:} \url{https://cloud.google.com/run/pricing}
    \item \textbf{Terraform GCP Provider:} \url{https://registry.terraform.io/providers/hashicorp/google/latest}
    \item \textbf{Robyn MMM:} \url{https://github.com/facebookexperimental/Robyn}
    \item \textbf{Streamlit Documentation:} \url{https://docs.streamlit.io}
\end{itemize}

\subsection{Getting Help}

For technical support:
\begin{enumerate}
    \item Review relevant documentation in the \texttt{docs/} directory
    \item Check Cloud Logging for error details
    \item Review GitHub repository issues
    \item Contact your implementation partner or technical lead
\end{enumerate}

\newpage

\section{Appendix A: Quick Reference}

\subsection{Cost Calculator}

Use this formula to estimate monthly costs:

\begin{equation}
\text{Monthly Cost} = \text{Fixed} + (\text{Jobs} \times \text{Cost per Job})
\end{equation}

Where:
\begin{itemize}
    \item Fixed = \$2-\$3/month
    \item Jobs = Number of training jobs per month
    \item Cost per Job = \$0.20 (benchmark) or \$1.33-\$2.00 (production)
\end{itemize}

\textbf{Example:} 100 production jobs/month = \$2 + (100 × \$1.67 avg) = \$169/month

\subsection{GCP Regions}

Recommended regions for deployment:
\begin{itemize}
    \item \texttt{europe-west1} (Belgium) - EU, low latency
    \item \texttt{europe-west4} (Netherlands) - EU, low latency
    \item \texttt{us-central1} (Iowa) - US, low cost
    \item \texttt{us-east1} (South Carolina) - US, low cost
\end{itemize}

Choose region closest to your Snowflake instance for optimal performance.

\subsection{Service Limits}

Default GCP quotas (can be increased via support request):
\begin{itemize}
    \item Cloud Run services per project: 100
    \item Cloud Run jobs per project: 1000
    \item Concurrent Cloud Run job executions: 100
    \item GCS buckets per project: 10,000
    \item Cloud Run request timeout: 60 minutes
\end{itemize}

\subsection{Glossary}

\begin{description}
    \item[Cloud Run] GCP's serverless container platform for web services and batch jobs
    \item[GCS] Google Cloud Storage - object storage service
    \item[IAM] Identity and Access Management - GCP's permission system
    \item[Robyn] Facebook's open-source Marketing Mix Modeling framework
    \item[Terraform] Infrastructure as Code tool for managing cloud resources
    \item[vCPU] Virtual CPU - compute allocation unit in cloud environments
    \item[Workload Identity] GCP's recommended way to authenticate from external services
\end{description}

\newpage

\section{Appendix B: Deployment Checklist}

Use this checklist for initial deployment:

\subsection{Pre-Deployment}
\begin{itemize}
    \item Install required tools (gcloud, terraform, docker, git)
    \item Create GCP project with billing enabled
    \item Enable required GCP APIs
    \item Create Terraform state bucket
    \item Create Artifact Registry repository
    \item Set up Workload Identity Federation for GitHub
    \item Create deployer service account with permissions
    \item Create output GCS bucket
    \item Configure Google OAuth (consent screen + credentials)
    \item Generate Snowflake RSA key pair (or obtain password)
    \item Configure GitHub repository secrets
\end{itemize}

\subsection{Configuration}
\begin{itemize}
    \item Update \texttt{infra/terraform/backend.tf} with state bucket
    \item Update \texttt{infra/terraform/envs/prod.tfvars} with project details
    \item Update \texttt{.github/workflows/ci.yml} with project ID and region
    \item Verify all environment variables in workflows
    \item Review and adjust resource allocations (CPU/memory)
\end{itemize}

\subsection{Deployment}
\begin{itemize}
    \item Push code to GitHub (main branch)
    \item Monitor GitHub Actions workflow
    \item Verify Docker images in Artifact Registry
    \item Verify Cloud Run service deployment
    \item Verify Cloud Run job registration
    \item Verify Cloud Scheduler jobs created
\end{itemize}

\subsection{Verification}
\begin{itemize}
    \item Access web application URL
    \item Test Google OAuth login
    \item Test Snowflake connection
    \item Test data query and preview
    \item Run test training job (200×3 iterations)
    \item Verify results in GCS bucket
    \item Check Cloud Logging for errors
    \item Verify IAM permissions are correct
    \item Set up billing alerts
    \item Document deployment details
\end{itemize}

\subsection{Post-Deployment}
\begin{itemize}
    \item Run benchmark training job (2000×5)
    \item Run production training job (10000×5)
    \item Verify all features work as expected
    \item Train team on platform usage
    \item Set up monitoring dashboards
    \item Configure alerting rules
    \item Document any customizations made
    \item Schedule first maintenance review
\end{itemize}

\end{document}
